\documentclass[]{book}

\usepackage[utf8]{inputenc}


\usepackage{chc-standard}

\usepackage{listingsutf8}
\usepackage{showexpl}

\usepackage{pdfpages}

\usepackage[german]{babel}

\lstset{
  numbers=left,
    numbersep=1em,
    numberstyle=\scriptsize,% to hide number lines
 %   frame=single,
  %  framesep=\fboxsep,% expands outward, cannot affect if frame=none
 %   framerule=\fboxrule,% expands outward, cannot affect if frame=none
   rframe={},
    rulecolor=\color{red},% cannot affect if frame=none
    xleftmargin=\dimexpr\fboxsep+\fboxrule\relax,
    xrightmargin=\dimexpr\fboxsep+\fboxrule\relax,
    breaklines=true,
    breakindent=0pt,
    tabsize=2,
    columns=flexible,
    language={[LaTeX]TeX},
    basicstyle=\small\ttfamily\hbox{},
    keywordstyle=\color{blue},
    backgroundcolor=\color{cyan!10},
  pos=r,
  width=0.5\linewidth,
  explpreset={},
  extendedchars=true,
}

\usepackage{tikz}

%% TODO: when I use article we ahve an option clash with hpyerref settings in chc-standard.sty and with other stuff


\begin{document}


%\chapter{Arbeitsumgebung}


%\chapter{Textsatz}


%\chapter{Sonderfälle}

\inputminted[]{tex}{example1.tex} 

\fbox{\includegraphics[scale=0.5]{build/example1.pdf}}


Fußnoten\footnote{werden so gesetzt}

\section{Tabellen}




\begin{minted}[]{tex}

asd
\end{minted}


\section{Abbildungen}



\section{Sourcecode}

\begin{minted}[
  framesep=2mm,
  baselinestretch=1.2,  % set it a bit looser
% bgcolor=lightgray,
%  fontsize=\footnotesize,
  linenos]
{python}
import numpy as np
    
def incmatrix(genl1,genl2):
    m = len(genl1)
    n = len(genl2)
    M = None #to become the incidence matrix
    VT = np.zeros((n*m,1), int)  #dummy variable
    
    #compute the bitwise xor matrix
    M1 = bitxormatrix(genl1)
    M2 = np.triu(bitxormatrix(genl2),1) 

    for i in range(m-1):
        for j in range(i+1, m):
            [r,c] = np.where(M2 == M1[i,j])
            for k in range(len(r)):
                VT[(i)*n + r[k]] = 1;
                VT[(i)*n + c[k]] = 1;
                VT[(j)*n + r[k]] = 1;
                VT[(j)*n + c[k]] = 1;
                
                if M is None:
                    M = np.copy(VT)
                else:
                    M = np.concatenate((M, VT), 1)
                
                VT = np.zeros((n*m,1), int)
    
    return M
\end{minted}


Standard einstellen:


\begin{minted}{latex}

\begin{document}


\end{document}

\end{minted}



\textbf{Quellen:}
\href{http://tug.ctan.org/macros/latex/contrib/minted/minted.pdf}{Referenzhandbuch zu minted}
\href{https://www.overleaf.com/learn/latex/Code_Highlighting_with_minted}{Overleaf Tutorial zu minted}



\section{Endliche Automaten}



\section{Sequenzdiagramme}




\section{Mathematischer Formelsatz}



\chapter{Textgrab}



IDEE: Ein side-by-side document machen mit geraden und ungeraden seiten wo man jeweils
den Latex source und das ergebnis enebeneinander sieht.

IDEE: Github repository zum eigenen üben. ebeno: overleaf zum eigenen üben.

IDEE. Zeichnungen ?    SVG inkludieren ? Gnuplot ?   Tikz ??

Dieses Dokument ist ein Musterdokument für \LaTeX.

Werkzeuge und Kompetenzen Liste:
* Docker
* Shell
* Git, Github, social coding
* node, php
* tex
* word
* 


asd




\end{document}





