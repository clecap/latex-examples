\documentclass{article}

\author{Maria Musterfrau}
\title{Mein erstes Papier}
\date{34. Februar 3021}

\begin{document}

\maketitle

\begin{abstract}
Hier steht ein kurzes Abstract. \LaTeX\ setzt es automatisch etwas kleiner, mit kleinen Einrückungen links und
rechts. Zu Beginn des Paragraphen sehen wir einen Einzug des Textes; der Paragraph ist auf der Seite zentriert.

Ein weiterer Paragraph wird durch eine Leerzeile begonnen, auch dieser Paragraph ist eingerückt. Es
gibt auch eine zweite Möglichkeit, einen neuen Paragraph zu beginnen. \par Bei dieser Technik
ist keine Leerzeile notwendig.
Eine Leerzeile ist aber ganz hilfreich, weil sie den Quelltext strukturiert und sie dem Autor damit
die Strukturierung seines Dokuments erleichtert. Deshalb verwende ich üblicherweise die Leerzeile.

\end{abstract}

\section{Einleitung}

Der volle Text der Einleitung beginnt auf der linken Seite und ohne Einzug. Alle weiteren
Paragraphen nach dem ersten beginnen aber mit einem Paragrapheneinzug -- schließlich ist der
erste Paragraph auch ganz deutlich zu erkennen.

Der Übergang zum zweiten Paragraphen ist also, wie hier, wieder durch einen
Einzug charakterisiert. Da ein Paragraph typischerweise länger ist als eine Zeile, macht das auch Sinn.
Bei extrem kurzen Paragraphen kann das dann aber auch mal komisch aussehen.

Wie hier (das ist ein Paragraph).

Wieder ein neuer Paragraph!

Jetzt noch ein Paragraph.

\hspace{-\parindent} Man kann diesen Einzug ganz leicht unterdrücken. Dazu fügt man einfach
einen negativen horizontalen Abstand in der Größe eines Paragrapheneinzugs ein, wie ich es hier gerade
gemacht habe, indem ich \textbackslash hspace\{-\textbackslash parindent\} eingefügt habe.
Es wird noch besser lesbar, wenn ich diesen Teil in einer anderen Font schreibe und ihn
dadurch als Code kennzeichne, also so: \texttt{\textbackslash hspace\{-\textbackslash parindent\}}.

Natürlich sind Sie jetzt neugierig geworden, wie ich das geschrieben habe, denn: Genau den Steuerbefehl,
der diesen Einzug unterdrückt hat, den kann ich ja nicht verwenden (Warum eigentlich?).
Sehen Sie sich das daher im Quelltext an!



\section{Hauptteil}

\parskip 0.25cm 
\parindent 0cm

Es ist auch möglich, den Paragrapheneinzug dauerhaft zu unterdrücken, Dazu ist es erforderlich,
den Paragrapheneinzug auf 0 zu setzen. Das hat dann aber den Nachteil, daß die einzelnen Paragraphen nicht
mehr voneinander getrennt sind. Dieses Problem läßt sich einfach lösen, indem man nun einen
vertikalen Abstand, ein sogenanntes skip einfügt.

Wie Sie gerade sehen, habe ich das zu Beginn dieses Abschnitts so gemacht. Persönlich gefällt mir
diese Einstellung auch am besten. Damit ich diese nicht immer selber setzen muß, habe ich einen 
sogenannten Style-File erstellt, in dem ich diese und viele weitere Voreinstellungen für meine Dokumente
in einer einzigen Zeile aktivieren kann.

\section{Schluß}

Wir kommen nun zum Schluß! Wir sind auf der zweiten Seite und sehen, daß auch die Seitennummerierung eine
automatische Voreinstellung von \LaTeX\ ist.

Wir können also bereits ein vollständiges \LaTeX-Dokument schreiben und erkennen, daß das System eine
Vielzahl von Formatierungsschritten automatisch für uns übernimmt. Ich empfehle Ihnen das aufmerksame
Studium des Quelltextes dieses Dokuments.


\end{document}
