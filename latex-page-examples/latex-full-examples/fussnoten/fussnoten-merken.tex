\documentclass[a5paper]{article}
\usepackage[pass]{geometry}

\usepackage[german]{babel}
\usepackage{blindtext}

\begin{document}

\newcounter{merker}  % Neuen Zähler anlegen zum speichern eines Fußnotenwertes.

\newcounter{beispiel}

Will man eine Fußnote\footnote{Diese hier!}\setcounter{merker}{\value{footnote}}
mehrfach verwenden\footnote{Insbesondere wenn dazwischen andere Fußnoten stehen},
dann ist es hilfreich, die Werte der Fußnote nicht als absolute Zahlen zu verwenden,
sondern sich die Fußnote in einer Variablen zwischenzuspeichern.

Man kann dann viel später\footnotemark[\value{merker}] auf den Wert\footnote{Den
der Zähler früher hatte} zurück%
\setcounter{beispiel}{\value{footnote}}%
greifen. Fügt man im Text später weitere Fußnoten\footnote{Noch eine!} ein,
dann verändert sich\footnotemark[\value{beispiel}] die absolute Zahl,
der Variablenname aber nicht. Es empfehlen sich sprechende Variablennamen.

Das Beispiel zeigt auch, wie die Befehle zum Speichern der Werte mitten im Text,
sogar mitten in einem Wort stehen können: Geeignete Kommentarzeichen verhindern,
daß das Zeilenende als Leerzeichen interpretiert wird. Das wäre näm
\setcounter{beispiel}{34}lich
sonst recht ärgerlich\footnote[\value{beispiel}]{Wie hier im Wort \textit{nämlich}!}.


\end{document}