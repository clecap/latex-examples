\documentclass[a5paper]{article}
\usepackage[pass]{geometry}

\usepackage[german]{babel}
\usepackage{blindtext}

\usepackage[multiple]{footmisc} % Fügt Kommata zwischen Fußnoten auf richtiger Höhe ein.

\begin{document}


Bei mehreren Fußnoten\footnote{Eins}\footnote{Zwei}\footnote{Drei}
sehen Kommata auf der Höhe der Fußnotenzeichen elegant aus.
Dazu bindet man das Paket \texttt{footmisc} mit der Option
\texttt{multiple} ein.

Das funktioniert aber nur, wenn \footnote{Vier} \footnote{Fünf} \footnote{Sechs}
zwischen den Fußnoten keine Leerzeichen stehen.

Zeilenenden zählen wie Leerzeichen\footnote{Sieben} 
\footnote{Acht}
\footnote{Neun}
zwischen den Fußnoten.


Zeilenenden zählen wie Leerzeichen\footnote{Sieben}% 
\footnote{Acht}%
\footnote{Neun}
zwischen den Fußnoten.

Vermutlich sieht es \footnote{zehn}%
\footnote{Elf}%
\footnote{Zwölf}
am Besten aus, wenn man bei mehreren Fußnoten einen
Abstand vor dem ersten Fußnotenzeichen beläßt.

\end{document}