\documentclass[a5paper]{article}
\usepackage[pass]{geometry}

\usepackage[german]{babel}

\usepackage{bigfoot}

\DeclareNewFootnote{default}
\DeclareNewFootnote{A}[alph]

\begin{document}

Eine weitere Möglichkeit des Pakets bigfoot\footnote{Gerne auch bei kritischen Ausgaben in
den Geisteswissenschaften genutzt.} sind verschachtelte Fußnoten%
\footnote{Das sind Fußnoten wie dieser hier, die dann selber wieder eine 
Fußnote\footnoteA{Diese hier} haben können}.

In den Naturwissenschaften gibt es anders als in den Geis\-teswissenschaften\footnoteA{und übrigens auch in
der Juristerei} jedoch nur selten Fußnoten.

Der letzte Absatz zeigt, daß Fußnotenebenen auch gemischt werden können.

\end{document}