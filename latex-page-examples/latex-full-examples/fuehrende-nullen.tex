\documentclass[a5paper]{article}
\usepackage[pass]{geometry}

\usepackage{fmtcount}%   Ein hilfreiches Paket für die Formatierung von Zahlen

\begin{document}

\newcounter{mycounter}%  Definiere neuen Zähler

\newcommand{\printcntr}{%                     Definiere neuen Befehl
  \stepcounter{mycounter}%                    Erhöhe Zähler
  \padzeroes[2]{\decimal{mycounter}}     &
  \padzeroes[4]{\binary{mycounter}}      &
  \padzeroes[3]{\octal{mycounter}}       &
  \padzeroes[3]{\HEXADecimal{mycounter}} %
}

\begin{tabular}{@{}cccc}
  \textbf{Dezimal} & \textbf{Binär} & \textbf{Oktal} & \textbf{Hexa} \\ \hline
  \printcntr \\ \printcntr \\ \printcntr \\ \printcntr \\ \printcntr \\
  \printcntr \\ \printcntr \\ \printcntr \\ \printcntr \\ \printcntr \\
  \printcntr \\ \printcntr \\ \printcntr \\ \printcntr \\ \printcntr \\
  \printcntr \\ \printcntr \\ \printcntr \\ \printcntr \\ \printcntr 
\end{tabular}  
\end{document}

