\documentclass{article}

\usepackage{../chc-exa-style}

\begin{document}

\note{Italic correction vermeidet die Kollision zwischen Zeichen,
typischerweise bei italic Schriften.}

\begin{example}
{\em f}B
\end{example}

\begin{example}
{\em f}\/B
\end{example}



\note{Slanted steht auch schief und dort ist italic correction 
ebenso ein Thema.}

\begin{example}
{\sl f}B
\end{example}

\begin{example}
{\sl f}\/B
\end{example}



\note{\texttt{\textbackslash textit}, \texttt{\textbackslash emph} 
und \texttt{\textbackslash testsl} haben die italic correction eingebaut.}

\begin{example}
\textit{f}B
\end{example}

\begin{example}
\textit{f}\/B
\end{example}

\begin{example}
\emph{f}B
\end{example}

\begin{example}
\emph{f}\/B
\end{example}

\begin{example}
\textsl{f}B
\end{example}

\begin{example}
\textsl{f}\/B
\end{example}


\note{Unterdrücken der Italic Correction:}

\begin{example}
\textit{\nocorr f}B
\end{example}

\begin{example}
\textit{f \nocorr}B
\end{example}


\note{\texttt{\textbackslash em}, 
\texttt{\textbackslash it},
\texttt{\textbackslash sl},
\texttt{\textbackslash itshape} 
und \texttt{\textbackslash slshape} benötigen italic correction.}

\begin{example}
{\em f}B
\end{example}

\begin{example}
{\em f}\/B
\end{example}

\begin{example}
{\it f}B
\end{example}

\begin{example}
{\it f}\/B
\end{example}

\begin{example}
{\sl f}B
\end{example}

\begin{example}
{\sl f}\/B
\end{example}

\begin{example}
{\itshape f}B
\end{example}

\begin{example}
{\itshape f}\/B
\end{example}

\begin{example}
{\slshape f}B
\end{example}

\begin{example}
{\slshape f}\/B
\end{example}

\note{Italic correction spielt auch bei non-italic
Schriften eine Rolle in der Kollisionsvermeidung:}

\begin{example}
pdf\TeX
\end{example}

\begin{example}
pdf\/\TeX
\end{example}






\end{document}