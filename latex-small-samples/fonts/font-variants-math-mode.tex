\documentclass{article}

\usepackage{../chc-exa-style}

\begin{document}

\begin{example}
$a=b$
\end{example}

\begin{example}
$\mathrm{a} = b$
\end{example}

\begin{example}
$\mathbf{a} = b$
\end{example}

\begin{example}
$\mathsf{a} = b$
\end{example}

\begin{example}
$\mathtt{a} = b$
\end{example}

\begin{example}
$\mathit{a} = b$
\end{example}

\begin{example}
$\mathnormal{a} = b$
\end{example}

\begin{example}
$\mathcal{a} = b$
\end{example}

\begin{example}
$\mathcal{A} = b$
\end{example}


\note{Textbefehle wirken kumulativ}

\begin{example}
\textbf{\textit{b}}
\end{example}


\note{Mathebefehle wirken nicht kumulativ:}

\begin{example}
$\mathbf{\mathit{a}} = b$
\end{example}

\note{Mathebefehle wirken nur auf das nächste Token:}

\begin{example}
$\mathbf a = b$
\end{example}

\note{Erforderlichenfalls mehrere Token durch Grupperierung zu einem Token zusammenfassen:}

\begin{example}
$\mathbf{a = b} =c$
\end{example}

\note{Komplettumstellung auf bold:}

\begin{example}
\mathversion{bold}$a = b$
\end{example}


\note{Komplettrückstellung auf normal:}

\begin{example}
\mathversion{bold}$a$\mathversion{normal}$a$
\end{example}

\note{Komplettrückstellung wirkt nur vor dem Mathemodus:}

\begin{example}
$\mathversion{bold}a$
\end{example}

Kalligraphische Fonts:

\begin{example}
$\mathcal{A} =\cal B$
\end{example}

Kalligraphische nur für Großbuchstaben
\begin{example}
$\mathcal{A} =\cal b$
\end{example}





\end{document}