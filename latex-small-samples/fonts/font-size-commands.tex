\documentclass[a4paper]{article}

\usepackage{../chc-exa-style}

\begin{document}


\begin{example}
\tiny{tiny}
\end{example}


\begin{example}
\scriptsize{scriptsize}
\end{example}


\begin{example}
\footnotesize{footnotesize}
\end{example}


\begin{example}
\small{small}
\end{example}


\begin{example}
\normalsize{normalsize}
\end{example}

\begin{example}
normalsize
\end{example}

\begin{example}
\large{large}
\end{example}


\begin{example}
\Large{Large}
\end{example}


\begin{example}
\LARGE{LARGE}
\end{example}


\begin{example}
\huge{huge}
\end{example}


\begin{example}
\Huge{Huge}
\end{example}

\note{Wie die folgenden zwei Vergleiche identischer Texte zeigen, 
beeinflussen die Befehle die Größe nicht linear, sondern verändern auch
die Form der Buchstaben. Die einzelnen Fonts verschiedener Größe sind separat entworfen 
und gehen nicht nur
durch einfache geometrische Skalierung auseinander hervor.}

\begin{example}
\tiny{Huge}
\end{example}

\begin{example}
\Huge{Huge}
\end{example}

\note{Größenangaben wirken ab der Angabe:}

\begin{example}
Normal \tiny tiny und \Large Large
\end{example}

\note{Größenangaben können auf Scopes beschränkt werden:}

\begin{example}
Normal {\tiny tiny} und {\Large Large}
\end{example}


\note{Größenangaben wirken im mathematischen Modus nicht:}

\begin{example}
$ {\tiny t} = {\large u} $
\end{example}

\note{Im mathematischen Modus können sie durch \texttt{\textbackslash mbox} eingeschmuggelt werden:}

\begin{example}
$ A_{\mbox{normal}} 
= B_{\mbox{\scriptsize klein}} $
\end{example}

\note{Will man in der mbox in der Font-Variante \textit{italics verbleiben}, was sinnvoll
erscheint, so nutzt man die einschlägigen Font-Kommandos für den Textmodus.
}

\begin{example}
$ A_{\mbox{\scriptsize \textit{normal}}} 
= B_{\mbox{\scriptsize \textit{klein}}} $
\end{example}


\note{Im Mathe-Modus aber wirken \texttt{\textbackslash scriptstyle} und \texttt{\textbackslash displaystyle}
auf die Größe; beide Einstellungen haben aber eine weitergehende Bedeutung, die sich
auch auf Abstände von Operatoren, die Basislinie der Formel und
die Struktur des Satzes beziehen.}

\begin{example}
$ {\scriptstyle t} = \displaystyle u} $
\end{example}


\begin{example}
$ \scriptstyle{t =} \displaystyle u $
\end{example}

\begin{example}
$ \scriptstyle \lim_{n\to\infty} = 
\displaystyle \lim_{n\to\infty} $
\end{example}

\pagebreak


\note{Größenangaben existieren auch als Environments:}

\begin{example}
Wie \begin{tiny}hier\end{tiny} etwa
\end{example}

\note{Bei mehrzeiliger Schreibweise von Umgebungen können dabei aber unerwünschte
Leerzeichen durch das Zeilenende auftreten:}
\vspace*{0.3cm}

\begin{example}
Wie 
\begin{tiny}
hier
\end{tiny}
etwa
\end{example}

\note{Diese unerwünschten Leerzeichen kann man durch ein Kommentarzeichen am Ende
unterdrücken, was aber umständlich ist. Daher werden Umgebungen zur Festlegung der Fontgröße 
eigentlich nicht verwendet.}
\vspace*{0.3cm}

\begin{example}
Wie 
\begin{tiny}%
hier
\end{tiny}
etwa
\end{example}

\vspace*{0.3cm}

\begin{example}
Wie 
\begin{tiny}%
hier
\end{tiny}%
etwa
\end{example}

\vspace*{0.3cm}

\begin{example}
Wie 
\begin{tiny}
hier
\end{tiny}%
etwa
\end{example}






\end{document}